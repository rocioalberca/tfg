\chapter{Planificación}

En el presente capítulo se describe la metodología de desarrollo empleada durante la creación del proyecto. Se reflejan las decisiones tomadas a lo largo del proceso, así como la forma en que se ha estructurado y organizado el trabajo. Todo ello con el propósito de facilitar el seguimiento del desarrollo, dejando constancia de los avances y de cómo se ha gestionado cada etapa del proyecto.
\section{Metodología}

La metodología empleada en este proyecto es la de \textbf{desarrollo ágil}. Esta consiste en realizar de manera iterativa pequeñas unidades de trabajo enfocadas a la satisfacción del cliente. En este caso, el cliente es el tribunal evaluador del Trabajo de Fin de Grado.

El enfoque de desarrollo ágil permite una evolución progresiva del proyecto. De esta manera, se verifica de forma continua que el trabajo realizado responde adecuadamente a las necesidades del cliente, lo que facilita detectar a tiempo desviaciones o cambios necesarios. A diferencia de los enfoques tradicionales, en los que la validación suele ocurrir al final del proceso, el desarrollo ágil permite una supervisión constante y mayor adaptación a lo largo del ciclo de vida del proyecto \cite{AgileAlliance2024}.

\section{Gestión del desarrollo con GitHub}

Como consecuencia de adoptar la metodología de desarrollo ágil, el desarrollo del proyecto debe basarse en un sistema de control. Se ha optado por usar un repositorio \textbf{GitHub} para documentar y seguir el proceso del proyecto. 

Para seguir las pautas de la metodología, se han definido \textbf{historias de usuario}, que reflejan las necesidades y expectativas del cliente en relación con el proyecto. Estas sirven como guía para priorizar y estructurar las tareas, y se han recogido en el repositorio del proyecto mediante \textit{issues} de GitHub. Cada \textit{issue} representa una tarea concreta cuyo cumplimiento satisface una historia de usuario.

La gestión del flujo de trabajo se ha llevado a cabo mediante un tablero \textbf{Kanban}, que se muestra en la Figura ~\ref{fig:kanban}. Integrado en GitHub Projects, permite visualizar el estado de cada tarea (pendiente, en progreso, en revisión y completada) y facilita el seguimiento continuo del proyecto \cite{GitHubProjects2024}. 

\vspace{10mm}

\begin{figure}[h]
    \centering
    \includegraphics[width=14cm]{Captura de pantalla Kanban.png}
    \caption{Tablero Kanban utilizado durante el desarrollo del TFG.}
    \label{fig:kanban}
\end{figure}

\vspace{10mm}

El desarrollo se ha organizado en \textbf{hitos} o, en GitHub, \textit{milestones}, que agrupan varias \textit{issues} en versiones concretas del proyecto. Un \textit{milestone} se trata de un producto mínimamente viable, y representa un estado alcanzable y verificable del desarrollo \cite{GitHubMilestones2024}.

La documentación mediante un repositorio de GitHub permite un seguimiento claro del avance del proyecto, así como mantener una organización en el mismo. Además, las \textit{issues} y los \textit{milestones} ayudan a asegurar y verificar que los requisitos definidos por el cliente se corresponden con el trabajo realizado. 

\section{Organización en milestones}

Todo el proceso de creación del proyecto ha sido subido, gestionado y documentado en el repositorio de {\textbf{GitHub}}: \url{https://github.com/rocalbbeb/tfg.git}

A partir del seguimiento realizado en el repositorio, este apartado recoge los milestones o productos mínimamente viables definidos durante el desarrollo. Cada uno de ellos representa un hito del trabajo que refleja el progreso alcanzado en las distintas fases del proyecto.

\href{https://github.com/rocalbbeb/tfg/milestone/1}{\textbf{Milestone 0}}

Primer hito del TFG: Implementación de la infraestructura básica, para que el proyecto compile y no tenga faltas de ortografía.

Descripción: 

Elaboración del producto inicial del TFG con LaTeX, con la infraestructura básica que asegure que el proyecto compile correctamente y no haya faltas de ortografía o gramaticales. Además, la creación de un Makefile para trabajar localmente con el proyecto. Y, por último, la explicación de la metodología seguida en el desarrollo del proyecto, así como los milestones y las historias de usuario.